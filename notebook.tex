
% Default to the notebook output style

    


% Inherit from the specified cell style.




    
\documentclass[11pt]{article}

    
    
    \usepackage[T1]{fontenc}
    % Nicer default font (+ math font) than Computer Modern for most use cases
    \usepackage{mathpazo}

    % Basic figure setup, for now with no caption control since it's done
    % automatically by Pandoc (which extracts ![](path) syntax from Markdown).
    \usepackage{graphicx}
    % We will generate all images so they have a width \maxwidth. This means
    % that they will get their normal width if they fit onto the page, but
    % are scaled down if they would overflow the margins.
    \makeatletter
    \def\maxwidth{\ifdim\Gin@nat@width>\linewidth\linewidth
    \else\Gin@nat@width\fi}
    \makeatother
    \let\Oldincludegraphics\includegraphics
    % Set max figure width to be 80% of text width, for now hardcoded.
    \renewcommand{\includegraphics}[1]{\Oldincludegraphics[width=.8\maxwidth]{#1}}
    % Ensure that by default, figures have no caption (until we provide a
    % proper Figure object with a Caption API and a way to capture that
    % in the conversion process - todo).
    \usepackage{caption}
    \DeclareCaptionLabelFormat{nolabel}{}
    \captionsetup{labelformat=nolabel}

    \usepackage{adjustbox} % Used to constrain images to a maximum size 
    \usepackage{xcolor} % Allow colors to be defined
    \usepackage{enumerate} % Needed for markdown enumerations to work
    \usepackage{geometry} % Used to adjust the document margins
    \usepackage{amsmath} % Equations
    \usepackage{amssymb} % Equations
    \usepackage{textcomp} % defines textquotesingle
    % Hack from http://tex.stackexchange.com/a/47451/13684:
    \AtBeginDocument{%
        \def\PYZsq{\textquotesingle}% Upright quotes in Pygmentized code
    }
    \usepackage{upquote} % Upright quotes for verbatim code
    \usepackage{eurosym} % defines \euro
    \usepackage[mathletters]{ucs} % Extended unicode (utf-8) support
    \usepackage[utf8x]{inputenc} % Allow utf-8 characters in the tex document
    \usepackage{fancyvrb} % verbatim replacement that allows latex
    \usepackage{grffile} % extends the file name processing of package graphics 
                         % to support a larger range 
    % The hyperref package gives us a pdf with properly built
    % internal navigation ('pdf bookmarks' for the table of contents,
    % internal cross-reference links, web links for URLs, etc.)
    \usepackage{hyperref}
    \usepackage{longtable} % longtable support required by pandoc >1.10
    \usepackage{booktabs}  % table support for pandoc > 1.12.2
    \usepackage[inline]{enumitem} % IRkernel/repr support (it uses the enumerate* environment)
    \usepackage[normalem]{ulem} % ulem is needed to support strikethroughs (\sout)
                                % normalem makes italics be italics, not underlines
    

    
    
    % Colors for the hyperref package
    \definecolor{urlcolor}{rgb}{0,.145,.698}
    \definecolor{linkcolor}{rgb}{.71,0.21,0.01}
    \definecolor{citecolor}{rgb}{.12,.54,.11}

    % ANSI colors
    \definecolor{ansi-black}{HTML}{3E424D}
    \definecolor{ansi-black-intense}{HTML}{282C36}
    \definecolor{ansi-red}{HTML}{E75C58}
    \definecolor{ansi-red-intense}{HTML}{B22B31}
    \definecolor{ansi-green}{HTML}{00A250}
    \definecolor{ansi-green-intense}{HTML}{007427}
    \definecolor{ansi-yellow}{HTML}{DDB62B}
    \definecolor{ansi-yellow-intense}{HTML}{B27D12}
    \definecolor{ansi-blue}{HTML}{208FFB}
    \definecolor{ansi-blue-intense}{HTML}{0065CA}
    \definecolor{ansi-magenta}{HTML}{D160C4}
    \definecolor{ansi-magenta-intense}{HTML}{A03196}
    \definecolor{ansi-cyan}{HTML}{60C6C8}
    \definecolor{ansi-cyan-intense}{HTML}{258F8F}
    \definecolor{ansi-white}{HTML}{C5C1B4}
    \definecolor{ansi-white-intense}{HTML}{A1A6B2}

    % commands and environments needed by pandoc snippets
    % extracted from the output of `pandoc -s`
    \providecommand{\tightlist}{%
      \setlength{\itemsep}{0pt}\setlength{\parskip}{0pt}}
    \DefineVerbatimEnvironment{Highlighting}{Verbatim}{commandchars=\\\{\}}
    % Add ',fontsize=\small' for more characters per line
    \newenvironment{Shaded}{}{}
    \newcommand{\KeywordTok}[1]{\textcolor[rgb]{0.00,0.44,0.13}{\textbf{{#1}}}}
    \newcommand{\DataTypeTok}[1]{\textcolor[rgb]{0.56,0.13,0.00}{{#1}}}
    \newcommand{\DecValTok}[1]{\textcolor[rgb]{0.25,0.63,0.44}{{#1}}}
    \newcommand{\BaseNTok}[1]{\textcolor[rgb]{0.25,0.63,0.44}{{#1}}}
    \newcommand{\FloatTok}[1]{\textcolor[rgb]{0.25,0.63,0.44}{{#1}}}
    \newcommand{\CharTok}[1]{\textcolor[rgb]{0.25,0.44,0.63}{{#1}}}
    \newcommand{\StringTok}[1]{\textcolor[rgb]{0.25,0.44,0.63}{{#1}}}
    \newcommand{\CommentTok}[1]{\textcolor[rgb]{0.38,0.63,0.69}{\textit{{#1}}}}
    \newcommand{\OtherTok}[1]{\textcolor[rgb]{0.00,0.44,0.13}{{#1}}}
    \newcommand{\AlertTok}[1]{\textcolor[rgb]{1.00,0.00,0.00}{\textbf{{#1}}}}
    \newcommand{\FunctionTok}[1]{\textcolor[rgb]{0.02,0.16,0.49}{{#1}}}
    \newcommand{\RegionMarkerTok}[1]{{#1}}
    \newcommand{\ErrorTok}[1]{\textcolor[rgb]{1.00,0.00,0.00}{\textbf{{#1}}}}
    \newcommand{\NormalTok}[1]{{#1}}
    
    % Additional commands for more recent versions of Pandoc
    \newcommand{\ConstantTok}[1]{\textcolor[rgb]{0.53,0.00,0.00}{{#1}}}
    \newcommand{\SpecialCharTok}[1]{\textcolor[rgb]{0.25,0.44,0.63}{{#1}}}
    \newcommand{\VerbatimStringTok}[1]{\textcolor[rgb]{0.25,0.44,0.63}{{#1}}}
    \newcommand{\SpecialStringTok}[1]{\textcolor[rgb]{0.73,0.40,0.53}{{#1}}}
    \newcommand{\ImportTok}[1]{{#1}}
    \newcommand{\DocumentationTok}[1]{\textcolor[rgb]{0.73,0.13,0.13}{\textit{{#1}}}}
    \newcommand{\AnnotationTok}[1]{\textcolor[rgb]{0.38,0.63,0.69}{\textbf{\textit{{#1}}}}}
    \newcommand{\CommentVarTok}[1]{\textcolor[rgb]{0.38,0.63,0.69}{\textbf{\textit{{#1}}}}}
    \newcommand{\VariableTok}[1]{\textcolor[rgb]{0.10,0.09,0.49}{{#1}}}
    \newcommand{\ControlFlowTok}[1]{\textcolor[rgb]{0.00,0.44,0.13}{\textbf{{#1}}}}
    \newcommand{\OperatorTok}[1]{\textcolor[rgb]{0.40,0.40,0.40}{{#1}}}
    \newcommand{\BuiltInTok}[1]{{#1}}
    \newcommand{\ExtensionTok}[1]{{#1}}
    \newcommand{\PreprocessorTok}[1]{\textcolor[rgb]{0.74,0.48,0.00}{{#1}}}
    \newcommand{\AttributeTok}[1]{\textcolor[rgb]{0.49,0.56,0.16}{{#1}}}
    \newcommand{\InformationTok}[1]{\textcolor[rgb]{0.38,0.63,0.69}{\textbf{\textit{{#1}}}}}
    \newcommand{\WarningTok}[1]{\textcolor[rgb]{0.38,0.63,0.69}{\textbf{\textit{{#1}}}}}
    
    
    % Define a nice break command that doesn't care if a line doesn't already
    % exist.
    \def\br{\hspace*{\fill} \\* }
    % Math Jax compatability definitions
    \def\gt{>}
    \def\lt{<}
    % Document parameters
    \title{Relat?rio - Adicionando a funcionalidade de Tanque}
    
    
    

    % Pygments definitions
    
\makeatletter
\def\PY@reset{\let\PY@it=\relax \let\PY@bf=\relax%
    \let\PY@ul=\relax \let\PY@tc=\relax%
    \let\PY@bc=\relax \let\PY@ff=\relax}
\def\PY@tok#1{\csname PY@tok@#1\endcsname}
\def\PY@toks#1+{\ifx\relax#1\empty\else%
    \PY@tok{#1}\expandafter\PY@toks\fi}
\def\PY@do#1{\PY@bc{\PY@tc{\PY@ul{%
    \PY@it{\PY@bf{\PY@ff{#1}}}}}}}
\def\PY#1#2{\PY@reset\PY@toks#1+\relax+\PY@do{#2}}

\expandafter\def\csname PY@tok@w\endcsname{\def\PY@tc##1{\textcolor[rgb]{0.73,0.73,0.73}{##1}}}
\expandafter\def\csname PY@tok@c\endcsname{\let\PY@it=\textit\def\PY@tc##1{\textcolor[rgb]{0.25,0.50,0.50}{##1}}}
\expandafter\def\csname PY@tok@cp\endcsname{\def\PY@tc##1{\textcolor[rgb]{0.74,0.48,0.00}{##1}}}
\expandafter\def\csname PY@tok@k\endcsname{\let\PY@bf=\textbf\def\PY@tc##1{\textcolor[rgb]{0.00,0.50,0.00}{##1}}}
\expandafter\def\csname PY@tok@kp\endcsname{\def\PY@tc##1{\textcolor[rgb]{0.00,0.50,0.00}{##1}}}
\expandafter\def\csname PY@tok@kt\endcsname{\def\PY@tc##1{\textcolor[rgb]{0.69,0.00,0.25}{##1}}}
\expandafter\def\csname PY@tok@o\endcsname{\def\PY@tc##1{\textcolor[rgb]{0.40,0.40,0.40}{##1}}}
\expandafter\def\csname PY@tok@ow\endcsname{\let\PY@bf=\textbf\def\PY@tc##1{\textcolor[rgb]{0.67,0.13,1.00}{##1}}}
\expandafter\def\csname PY@tok@nb\endcsname{\def\PY@tc##1{\textcolor[rgb]{0.00,0.50,0.00}{##1}}}
\expandafter\def\csname PY@tok@nf\endcsname{\def\PY@tc##1{\textcolor[rgb]{0.00,0.00,1.00}{##1}}}
\expandafter\def\csname PY@tok@nc\endcsname{\let\PY@bf=\textbf\def\PY@tc##1{\textcolor[rgb]{0.00,0.00,1.00}{##1}}}
\expandafter\def\csname PY@tok@nn\endcsname{\let\PY@bf=\textbf\def\PY@tc##1{\textcolor[rgb]{0.00,0.00,1.00}{##1}}}
\expandafter\def\csname PY@tok@ne\endcsname{\let\PY@bf=\textbf\def\PY@tc##1{\textcolor[rgb]{0.82,0.25,0.23}{##1}}}
\expandafter\def\csname PY@tok@nv\endcsname{\def\PY@tc##1{\textcolor[rgb]{0.10,0.09,0.49}{##1}}}
\expandafter\def\csname PY@tok@no\endcsname{\def\PY@tc##1{\textcolor[rgb]{0.53,0.00,0.00}{##1}}}
\expandafter\def\csname PY@tok@nl\endcsname{\def\PY@tc##1{\textcolor[rgb]{0.63,0.63,0.00}{##1}}}
\expandafter\def\csname PY@tok@ni\endcsname{\let\PY@bf=\textbf\def\PY@tc##1{\textcolor[rgb]{0.60,0.60,0.60}{##1}}}
\expandafter\def\csname PY@tok@na\endcsname{\def\PY@tc##1{\textcolor[rgb]{0.49,0.56,0.16}{##1}}}
\expandafter\def\csname PY@tok@nt\endcsname{\let\PY@bf=\textbf\def\PY@tc##1{\textcolor[rgb]{0.00,0.50,0.00}{##1}}}
\expandafter\def\csname PY@tok@nd\endcsname{\def\PY@tc##1{\textcolor[rgb]{0.67,0.13,1.00}{##1}}}
\expandafter\def\csname PY@tok@s\endcsname{\def\PY@tc##1{\textcolor[rgb]{0.73,0.13,0.13}{##1}}}
\expandafter\def\csname PY@tok@sd\endcsname{\let\PY@it=\textit\def\PY@tc##1{\textcolor[rgb]{0.73,0.13,0.13}{##1}}}
\expandafter\def\csname PY@tok@si\endcsname{\let\PY@bf=\textbf\def\PY@tc##1{\textcolor[rgb]{0.73,0.40,0.53}{##1}}}
\expandafter\def\csname PY@tok@se\endcsname{\let\PY@bf=\textbf\def\PY@tc##1{\textcolor[rgb]{0.73,0.40,0.13}{##1}}}
\expandafter\def\csname PY@tok@sr\endcsname{\def\PY@tc##1{\textcolor[rgb]{0.73,0.40,0.53}{##1}}}
\expandafter\def\csname PY@tok@ss\endcsname{\def\PY@tc##1{\textcolor[rgb]{0.10,0.09,0.49}{##1}}}
\expandafter\def\csname PY@tok@sx\endcsname{\def\PY@tc##1{\textcolor[rgb]{0.00,0.50,0.00}{##1}}}
\expandafter\def\csname PY@tok@m\endcsname{\def\PY@tc##1{\textcolor[rgb]{0.40,0.40,0.40}{##1}}}
\expandafter\def\csname PY@tok@gh\endcsname{\let\PY@bf=\textbf\def\PY@tc##1{\textcolor[rgb]{0.00,0.00,0.50}{##1}}}
\expandafter\def\csname PY@tok@gu\endcsname{\let\PY@bf=\textbf\def\PY@tc##1{\textcolor[rgb]{0.50,0.00,0.50}{##1}}}
\expandafter\def\csname PY@tok@gd\endcsname{\def\PY@tc##1{\textcolor[rgb]{0.63,0.00,0.00}{##1}}}
\expandafter\def\csname PY@tok@gi\endcsname{\def\PY@tc##1{\textcolor[rgb]{0.00,0.63,0.00}{##1}}}
\expandafter\def\csname PY@tok@gr\endcsname{\def\PY@tc##1{\textcolor[rgb]{1.00,0.00,0.00}{##1}}}
\expandafter\def\csname PY@tok@ge\endcsname{\let\PY@it=\textit}
\expandafter\def\csname PY@tok@gs\endcsname{\let\PY@bf=\textbf}
\expandafter\def\csname PY@tok@gp\endcsname{\let\PY@bf=\textbf\def\PY@tc##1{\textcolor[rgb]{0.00,0.00,0.50}{##1}}}
\expandafter\def\csname PY@tok@go\endcsname{\def\PY@tc##1{\textcolor[rgb]{0.53,0.53,0.53}{##1}}}
\expandafter\def\csname PY@tok@gt\endcsname{\def\PY@tc##1{\textcolor[rgb]{0.00,0.27,0.87}{##1}}}
\expandafter\def\csname PY@tok@err\endcsname{\def\PY@bc##1{\setlength{\fboxsep}{0pt}\fcolorbox[rgb]{1.00,0.00,0.00}{1,1,1}{\strut ##1}}}
\expandafter\def\csname PY@tok@kc\endcsname{\let\PY@bf=\textbf\def\PY@tc##1{\textcolor[rgb]{0.00,0.50,0.00}{##1}}}
\expandafter\def\csname PY@tok@kd\endcsname{\let\PY@bf=\textbf\def\PY@tc##1{\textcolor[rgb]{0.00,0.50,0.00}{##1}}}
\expandafter\def\csname PY@tok@kn\endcsname{\let\PY@bf=\textbf\def\PY@tc##1{\textcolor[rgb]{0.00,0.50,0.00}{##1}}}
\expandafter\def\csname PY@tok@kr\endcsname{\let\PY@bf=\textbf\def\PY@tc##1{\textcolor[rgb]{0.00,0.50,0.00}{##1}}}
\expandafter\def\csname PY@tok@bp\endcsname{\def\PY@tc##1{\textcolor[rgb]{0.00,0.50,0.00}{##1}}}
\expandafter\def\csname PY@tok@fm\endcsname{\def\PY@tc##1{\textcolor[rgb]{0.00,0.00,1.00}{##1}}}
\expandafter\def\csname PY@tok@vc\endcsname{\def\PY@tc##1{\textcolor[rgb]{0.10,0.09,0.49}{##1}}}
\expandafter\def\csname PY@tok@vg\endcsname{\def\PY@tc##1{\textcolor[rgb]{0.10,0.09,0.49}{##1}}}
\expandafter\def\csname PY@tok@vi\endcsname{\def\PY@tc##1{\textcolor[rgb]{0.10,0.09,0.49}{##1}}}
\expandafter\def\csname PY@tok@vm\endcsname{\def\PY@tc##1{\textcolor[rgb]{0.10,0.09,0.49}{##1}}}
\expandafter\def\csname PY@tok@sa\endcsname{\def\PY@tc##1{\textcolor[rgb]{0.73,0.13,0.13}{##1}}}
\expandafter\def\csname PY@tok@sb\endcsname{\def\PY@tc##1{\textcolor[rgb]{0.73,0.13,0.13}{##1}}}
\expandafter\def\csname PY@tok@sc\endcsname{\def\PY@tc##1{\textcolor[rgb]{0.73,0.13,0.13}{##1}}}
\expandafter\def\csname PY@tok@dl\endcsname{\def\PY@tc##1{\textcolor[rgb]{0.73,0.13,0.13}{##1}}}
\expandafter\def\csname PY@tok@s2\endcsname{\def\PY@tc##1{\textcolor[rgb]{0.73,0.13,0.13}{##1}}}
\expandafter\def\csname PY@tok@sh\endcsname{\def\PY@tc##1{\textcolor[rgb]{0.73,0.13,0.13}{##1}}}
\expandafter\def\csname PY@tok@s1\endcsname{\def\PY@tc##1{\textcolor[rgb]{0.73,0.13,0.13}{##1}}}
\expandafter\def\csname PY@tok@mb\endcsname{\def\PY@tc##1{\textcolor[rgb]{0.40,0.40,0.40}{##1}}}
\expandafter\def\csname PY@tok@mf\endcsname{\def\PY@tc##1{\textcolor[rgb]{0.40,0.40,0.40}{##1}}}
\expandafter\def\csname PY@tok@mh\endcsname{\def\PY@tc##1{\textcolor[rgb]{0.40,0.40,0.40}{##1}}}
\expandafter\def\csname PY@tok@mi\endcsname{\def\PY@tc##1{\textcolor[rgb]{0.40,0.40,0.40}{##1}}}
\expandafter\def\csname PY@tok@il\endcsname{\def\PY@tc##1{\textcolor[rgb]{0.40,0.40,0.40}{##1}}}
\expandafter\def\csname PY@tok@mo\endcsname{\def\PY@tc##1{\textcolor[rgb]{0.40,0.40,0.40}{##1}}}
\expandafter\def\csname PY@tok@ch\endcsname{\let\PY@it=\textit\def\PY@tc##1{\textcolor[rgb]{0.25,0.50,0.50}{##1}}}
\expandafter\def\csname PY@tok@cm\endcsname{\let\PY@it=\textit\def\PY@tc##1{\textcolor[rgb]{0.25,0.50,0.50}{##1}}}
\expandafter\def\csname PY@tok@cpf\endcsname{\let\PY@it=\textit\def\PY@tc##1{\textcolor[rgb]{0.25,0.50,0.50}{##1}}}
\expandafter\def\csname PY@tok@c1\endcsname{\let\PY@it=\textit\def\PY@tc##1{\textcolor[rgb]{0.25,0.50,0.50}{##1}}}
\expandafter\def\csname PY@tok@cs\endcsname{\let\PY@it=\textit\def\PY@tc##1{\textcolor[rgb]{0.25,0.50,0.50}{##1}}}

\def\PYZbs{\char`\\}
\def\PYZus{\char`\_}
\def\PYZob{\char`\{}
\def\PYZcb{\char`\}}
\def\PYZca{\char`\^}
\def\PYZam{\char`\&}
\def\PYZlt{\char`\<}
\def\PYZgt{\char`\>}
\def\PYZsh{\char`\#}
\def\PYZpc{\char`\%}
\def\PYZdl{\char`\$}
\def\PYZhy{\char`\-}
\def\PYZsq{\char`\'}
\def\PYZdq{\char`\"}
\def\PYZti{\char`\~}
% for compatibility with earlier versions
\def\PYZat{@}
\def\PYZlb{[}
\def\PYZrb{]}
\makeatother


    % Exact colors from NB
    \definecolor{incolor}{rgb}{0.0, 0.0, 0.5}
    \definecolor{outcolor}{rgb}{0.545, 0.0, 0.0}



    
    % Prevent overflowing lines due to hard-to-break entities
    \sloppy 
    % Setup hyperref package
    \hypersetup{
      breaklinks=true,  % so long urls are correctly broken across lines
      colorlinks=true,
      urlcolor=urlcolor,
      linkcolor=linkcolor,
      citecolor=citecolor,
      }
    % Slightly bigger margins than the latex defaults
    
    \geometry{verbose,tmargin=1in,bmargin=1in,lmargin=1in,rmargin=1in}
    
    

    \begin{document}
    
    
    \maketitle
    
    

    
    \section{Relatório Mestrado}\label{relatuxf3rio-mestrado}

\textbf{Orientador}: Vandilberto Pereira Pinto

\textbf{Mestrando}: Killdary Aguiar de Santana

    \subsection{Introdução}\label{introduuxe7uxe3o}

A dissertação do mestrado propõe um Algorítmo Genético que solucione o
problema do Caixeiro Viajante. O algorítmo até a sua apresentação deverá
possuir as seguintes funcionalidades:

\begin{itemize}
\tightlist
\item
  \sout{Cálculo da menor rota};
\item
  Criar funcionalidade de tanque, de modo a percorrer a rota e voltar ao
  ponto inicial sempre que o tanque estiver perto de esvaziar;
\item
  Múltiplos pontos de carregamento, o algoritmo devrá redirecionar o
  robô ao ponto de carregamento mais próximo;
\item
  Adicionar peso aos pontos da rota, assim o algoritmo criara a rota com
  base nos pesoas e não distância;
\item
  Divisão de rotas para múltiplos robôs;
\item
  Integração com o Simulador VREP;
\item
  Construção do Robô e integração com o algorítmo;
\item
  Rotas com 3 dimenções, apara rotas de Drones;
\item
  Adaptar solução ao Matlab e ao Scilab;
\item
  Início da dissertação;
\end{itemize}

\begin{quote}
Todas as etapas da pesquisa estão listadas acima, a acada nova
funcionalidade desenvolvida será riscada representando como concluída.
Novas funcionalidades só poderão ser acrescentadas caso o orientador
aprove.
\end{quote}

O algorímo será desenvolvida em \textbf{Pyton}, esta linguagem foi
escolhida devido sua semelhanção ao \emph{Matlab} e possibilita a
encapsulação diretamente em um robô.

    \subsection{Funcionamento}\label{funcionamento}

O algorítmo foi desenvolvido utilizando orientação a objetos(OO). A
escolha da OO facilita a tarefa de acrescentar funcionalidades,
facilitando sua manutenção e atualização.

\subsection{Dependências}\label{dependuxeancias}

O algoritimo utiliza de 2 bibliotecas em seu funcionamento:

\begin{itemize}
\tightlist
\item
  \textbf{Numpy}: Biblioteca matemática do python, sua sintaxe e
  funcionalidades foram baseadas no Matlab;
\item
  \textbf{Matplotlib}: Biblioteca gpara gerar gráficos no python, será
  utilizada para representar a rota seguida pelo robô
\end{itemize}

\begin{quote}
O Matlab já possui todas as funcionalidades das bibliotecas citadas
acima
\end{quote}

    \subsection{Funcionalidade Tanque}\label{funcionalidade-tanque}

Segundo o cronograma a próxima funcionalidade a ser desenvolvida será o
acrescimo do tanque ao algorítimo. Para este propósito foi desenvolvido
um novo método que recebe a rota calculada pelo algoritmo junto com o
valor da distância máxima que o robô pode percorrer até a próxima
recarga.

Primeiramente será calculado uma rota para 50 pontos as serem visitados.
O algorítimo exige os seguintes parâmetros para seu funcionamento: *
\textbf{generation}: número de gerações que serão criadas pelo
algorítmo; * \textbf{population}: número de indivíduos que serão gerados
a cada geração; * \textbf{towns}: um arquivo de texto com todos os
pontos a serem visitados, cada linha do arquivo deve ser uma cidade.
Cada linha possui dois números que serão a localização do ponto em um
plano cartesiano;

Abaixo será criado uma rota para os 50 pontos:

    \begin{Verbatim}[commandchars=\\\{\}]
{\color{incolor}In [{\color{incolor}10}]:} \PY{k+kn}{from} \PY{n+nn}{AG\PYZus{}routes} \PY{k}{import} \PY{n}{CalulateRoutesTSP}
         \PY{k+kn}{import} \PY{n+nn}{numpy} \PY{k}{as} \PY{n+nn}{np}
         
         \PY{n}{calc\PYZus{}rota} \PY{o}{=} \PY{n}{CalulateRoutesTSP}\PY{p}{(}\PY{p}{)}
         
         \PY{n}{custo\PYZus{}rota}\PY{p}{,} \PY{n}{rota} \PY{o}{=} \PY{n}{calc\PYZus{}rota}\PY{o}{.}\PY{n}{ga}\PY{p}{(}\PY{n}{generation}\PY{o}{=}\PY{l+m+mi}{2500}\PY{p}{,}
                                         \PY{n}{population}\PY{o}{=}\PY{l+m+mi}{50}\PY{p}{,}
                                         \PY{n}{towns}\PY{o}{=}\PY{l+s+s1}{\PYZsq{}}\PY{l+s+s1}{./pontos.txt}\PY{l+s+s1}{\PYZsq{}}\PY{p}{)}
         
         \PY{n+nb}{print}\PY{p}{(}\PY{l+s+s1}{\PYZsq{}}\PY{l+s+s1}{Custo da menor Rota: }\PY{l+s+si}{\PYZob{}0\PYZcb{}}\PY{l+s+s1}{ }\PY{l+s+se}{\PYZbs{}n}\PY{l+s+s1}{Menor Rota:}\PY{l+s+si}{\PYZob{}1\PYZcb{}}\PY{l+s+s1}{\PYZsq{}}\PY{o}{.}\PY{n}{format}\PY{p}{(}\PY{n}{custo\PYZus{}rota}\PY{p}{,} \PY{n}{rota}\PY{p}{)}\PY{p}{)}
\end{Verbatim}


    \begin{Verbatim}[commandchars=\\\{\}]
Custo da menor Rota: 59.666246407315995 
Menor Rota:[  9.  20.   3.  42.  26.  14.  29.  17.   2.  18.   1.  44.   4.  21.  38.
  34.  28.  41.  25.  12.  45.  37.  13.  46.   5.  15.  33.  31.  11.   8.
  10.  27.   7.  16.  32.  19.  40.  47.  30.  36.  48.  39.   6.  23.   0.
  22.  43.  49.  24.  35.   9.]

    \end{Verbatim}

    Gráfico apresentando a rota acima:

    \begin{Verbatim}[commandchars=\\\{\}]
{\color{incolor}In [{\color{incolor}5}]:} \PY{n}{calc\PYZus{}rota}\PY{o}{.}\PY{n}{plota\PYZus{}rotas}\PY{p}{(}\PY{n}{calc\PYZus{}rota}\PY{o}{.}\PY{n}{mapa}\PY{p}{,} \PY{n}{rota}\PY{p}{)}
\end{Verbatim}


    \begin{center}
    \adjustimage{max size={0.9\linewidth}{0.9\paperheight}}{output_6_0.png}
    \end{center}
    { \hspace*{\fill} \\}
    
    Em posse da rota basta utilizar o método \texttt{tanque\_rota}, esse
método recebe os seguintes parâmetros:

\begin{itemize}
\tightlist
\item
  \textbf{rota}: rota que será seguida pelo robô;
\item
  \textbf{tanque}: valor que representa a autonomia do robô antes de uma
  nova recarga;
\end{itemize}

\begin{quote}
Até o momento o tanque representa a distância que o robô consegue se
mover.
\end{quote}

O algorítmo percorre a rota informada baseado no consumo do tanque,
quando o tanque esteja próximo de acabar o mesmo redireciona o robô ao
ponto de recarga, que neste caso será o ponto inicial.

O método \texttt{tanque\_rota} realiza as seguintes etapas:

\begin{enumerate}
\def\labelenumi{\arabic{enumi}.}
\tightlist
\item
  Saindo do ponto inicial é verificado se o robô possui tanque
  suficiente para ir para a próxima cidade e se o mesmo consegue voltar
  ao ponto de recarga. Essa ação é necessária para determinar se o robô
  consegue voltar ao pono de recarga;
\item
  Caso o robô consiga ir apenas ao próximo ponto e não consiga ir ao
  ponto de recarga, ou ainda na localização atual ele não tenha energia
  suficiente para chegar ao próximo ponto, o mesmo é enviado ao ponto de
  recarga;
\item
  Caso o robô não possua carga para retornar ao ponto de recarga, o
  mesmo estaciona no último ponto visitado;
\end{enumerate}

Abaixo a utilização do método \texttt{tanque\ rota} com 3 valores de
tanques e seus respectivos gráficos:

\subsubsection{Tanque com valor 10}\label{tanque-com-valor-10}

    \begin{Verbatim}[commandchars=\\\{\}]
{\color{incolor}In [{\color{incolor}7}]:} \PY{n}{custo\PYZus{}rota\PYZus{}tanque\PYZus{}10}\PY{p}{,} \PY{n}{rota\PYZus{}tanque\PYZus{}10} \PY{o}{=} \PY{n}{calc\PYZus{}rota}\PY{o}{.}\PY{n}{tanque\PYZus{}rota}\PY{p}{(}\PY{n}{rota}\PY{p}{,} \PY{l+m+mi}{10}\PY{p}{)}
        
        \PY{n+nb}{print}\PY{p}{(}\PY{l+s+s1}{\PYZsq{}}\PY{l+s+s1}{Custo: }\PY{l+s+si}{\PYZob{}0\PYZcb{}}\PY{l+s+s1}{ }\PY{l+s+se}{\PYZbs{}n}\PY{l+s+s1}{Rota:}\PY{l+s+si}{\PYZob{}1\PYZcb{}}\PY{l+s+s1}{\PYZsq{}}\PY{o}{.}\PY{n}{format}\PY{p}{(}\PY{n}{custo\PYZus{}rota\PYZus{}tanque\PYZus{}10}\PY{p}{,} \PY{n}{rota\PYZus{}tanque\PYZus{}10}\PY{p}{)}\PY{p}{)}
\end{Verbatim}


    \begin{Verbatim}[commandchars=\\\{\}]
5.62249062953 5.62249062953 0.0 10
Custo: 18.878764698238054 
Rota:[43, 0, 23, 6, 39, 48, 43, 36, 30, 43]

    \end{Verbatim}

    \begin{Verbatim}[commandchars=\\\{\}]
{\color{incolor}In [{\color{incolor}11}]:} \PY{n}{calc\PYZus{}rota}\PY{o}{.}\PY{n}{plota\PYZus{}rotas}\PY{p}{(}\PY{n}{calc\PYZus{}rota}\PY{o}{.}\PY{n}{mapa}\PY{p}{,} \PY{n}{np}\PY{o}{.}\PY{n}{array}\PY{p}{(}\PY{n}{rota\PYZus{}tanque\PYZus{}10}\PY{p}{,} \PY{n}{dtype}\PY{o}{=}\PY{n}{np}\PY{o}{.}\PY{n}{int32}\PY{p}{)}\PY{p}{)}
\end{Verbatim}


    \begin{center}
    \adjustimage{max size={0.9\linewidth}{0.9\paperheight}}{output_9_0.png}
    \end{center}
    { \hspace*{\fill} \\}
    
    Na rota calculada acima o algoritmo só percorreu 10 pontos, já que foi
determinado que não há tanque para visitar a próxima cidade, assim ele
coloca o agente no ponto de recarga, para preservá-lo.

    \subsubsection{Tanque com valor 20}\label{tanque-com-valor-20}

    \begin{Verbatim}[commandchars=\\\{\}]
{\color{incolor}In [{\color{incolor}12}]:} \PY{n}{custo\PYZus{}rota\PYZus{}tanque\PYZus{}20}\PY{p}{,} \PY{n}{rota\PYZus{}tanque\PYZus{}20} \PY{o}{=} \PY{n}{calc\PYZus{}rota}\PY{o}{.}\PY{n}{tanque\PYZus{}rota}\PY{p}{(}\PY{n}{rota}\PY{p}{,} \PY{l+m+mi}{20}\PY{p}{)}
         
         \PY{n+nb}{print}\PY{p}{(}\PY{l+s+s1}{\PYZsq{}}\PY{l+s+s1}{Custo: }\PY{l+s+si}{\PYZob{}0\PYZcb{}}\PY{l+s+s1}{ }\PY{l+s+se}{\PYZbs{}n}\PY{l+s+s1}{Rota:}\PY{l+s+si}{\PYZob{}1\PYZcb{}}\PY{l+s+s1}{\PYZsq{}}\PY{o}{.}\PY{n}{format}\PY{p}{(}\PY{n}{custo\PYZus{}rota\PYZus{}tanque\PYZus{}20}\PY{p}{,} \PY{n}{rota\PYZus{}tanque\PYZus{}20}\PY{p}{)}\PY{p}{)}
\end{Verbatim}


    \begin{Verbatim}[commandchars=\\\{\}]
10.3287667332 10.3287667332 0.0 20
Custo: 116.2766145723432 
Rota:[9, 20, 3, 42, 26, 14, 29, 17, 2, 18, 1, 44, 4, 21, 38, 34, 28, 9, 41, 25, 12, 45, 37, 13, 46, 5, 9, 15, 33, 9, 31, 9, 11, 8, 9, 10, 9]

    \end{Verbatim}

    \begin{Verbatim}[commandchars=\\\{\}]
{\color{incolor}In [{\color{incolor}13}]:} \PY{n}{calc\PYZus{}rota}\PY{o}{.}\PY{n}{plota\PYZus{}rotas}\PY{p}{(}\PY{n}{calc\PYZus{}rota}\PY{o}{.}\PY{n}{mapa}\PY{p}{,} \PY{n}{np}\PY{o}{.}\PY{n}{array}\PY{p}{(}\PY{n}{rota\PYZus{}tanque\PYZus{}20}\PY{p}{,} \PY{n}{dtype}\PY{o}{=}\PY{n}{np}\PY{o}{.}\PY{n}{int32}\PY{p}{)}\PY{p}{)}
\end{Verbatim}


    \begin{center}
    \adjustimage{max size={0.9\linewidth}{0.9\paperheight}}{output_13_0.png}
    \end{center}
    { \hspace*{\fill} \\}
    
    Na rota com tanque valor 20 fram visitadas mais cidades, porém ainda
assim existe distâncias que o algoritimo determinou que não possui
tanque suficiente para sair do ponto de recarga e voltar ao mesmos,
sendo assim o agente é colocado no ponto de recarga, para preservá-lo.

    \subsubsection{Tanque com valor 30}\label{tanque-com-valor-30}

    \begin{Verbatim}[commandchars=\\\{\}]
{\color{incolor}In [{\color{incolor}17}]:} \PY{n}{custo\PYZus{}rota\PYZus{}tanque\PYZus{}30}\PY{p}{,} \PY{n}{rota\PYZus{}tanque\PYZus{}30} \PY{o}{=} \PY{n}{calc\PYZus{}rota}\PY{o}{.}\PY{n}{tanque\PYZus{}rota}\PY{p}{(}\PY{n}{rota}\PY{p}{,} \PY{l+m+mi}{30}\PY{p}{)}
         
         \PY{n+nb}{print}\PY{p}{(}\PY{l+s+s1}{\PYZsq{}}\PY{l+s+s1}{Custo: }\PY{l+s+si}{\PYZob{}0\PYZcb{}}\PY{l+s+s1}{ }\PY{l+s+se}{\PYZbs{}n}\PY{l+s+s1}{Rota:}\PY{l+s+si}{\PYZob{}1\PYZcb{}}\PY{l+s+s1}{\PYZsq{}}\PY{o}{.}\PY{n}{format}\PY{p}{(}\PY{n}{custo\PYZus{}rota\PYZus{}tanque\PYZus{}30}\PY{p}{,} \PY{n}{rota\PYZus{}tanque\PYZus{}30}\PY{p}{)}\PY{p}{)}
\end{Verbatim}


    \begin{Verbatim}[commandchars=\\\{\}]
0.0 0.0 0.0 30
Custo: 108.5032409026573 
Rota:[9, 20, 3, 42, 26, 14, 29, 17, 2, 18, 1, 44, 4, 21, 38, 34, 28, 41, 25, 12, 45, 9, 37, 13, 46, 5, 15, 33, 31, 11, 8, 10, 27, 7, 9, 16, 32, 19, 40, 47, 30, 36, 9, 48, 39, 6, 23, 0, 22, 43, 49, 24, 35, 9]

    \end{Verbatim}

    \begin{Verbatim}[commandchars=\\\{\}]
{\color{incolor}In [{\color{incolor}18}]:} \PY{n}{calc\PYZus{}rota}\PY{o}{.}\PY{n}{plota\PYZus{}rotas}\PY{p}{(}\PY{n}{calc\PYZus{}rota}\PY{o}{.}\PY{n}{mapa}\PY{p}{,} \PY{n}{np}\PY{o}{.}\PY{n}{array}\PY{p}{(}\PY{n}{rota\PYZus{}tanque\PYZus{}30}\PY{p}{,} \PY{n}{dtype}\PY{o}{=}\PY{n}{np}\PY{o}{.}\PY{n}{int32}\PY{p}{)}\PY{p}{)}
\end{Verbatim}


    \begin{center}
    \adjustimage{max size={0.9\linewidth}{0.9\paperheight}}{output_17_0.png}
    \end{center}
    { \hspace*{\fill} \\}
    
    Com o valor 30 o algoritimo conseguiu traçar uma rota que percorreu
todos os pontos, visitando o ponto de recarga 4 vezes no total.

\subsection{Conclusão}\label{conclusuxe3o}

A funcionalidade consegue calcular a rota, baseada em um tanque, porém
para um melhor aproveitamento será necessário adicionar multiplos pontos
de recarga.


    % Add a bibliography block to the postdoc
    
    
    
    \end{document}
